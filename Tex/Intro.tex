\chapter{绪论}\label{chap:Intro}

\section{研究背景与意义}
激光非视域成像技术用来观察路口拐角、有门窗的房间等不便进入的场合,能够绕过遮挡,实现视线以外区域目标的定位与成像(图1.1)[1],有很好的应用前景,可以应用于驾驶辅助可以有效预防交通事故、在自动驾驶中有助于危险规避;

应用于反恐作战,可以提升作战中的侦查能力,掌握主动性;

还可以用于定位灾难救援中被困人员,有效施救。

未来更高精度的研究还可能用于古迹发掘甚至是医学诊疗等领域。

 
图1.1  非视域成像技术应用场景
Figure 1.1 Application of NLOS imaging technology

在这些应用中,由于目标不在观察者可见范围,传统成像方法对此束手无策[2],而非视域成像技术可以通过测量经过数次反射的返回信号,再用反演算法进行重构。
研究相关理论和影响因素将对以后研究起到一定的辅助作用。

非视域成像过程中,激光经过了多次漫反射,损失了空间信息和绝大部分的强度。
因而成像过程对数据收集有很高的要求,只有精度很高的数据,才能重建出可辨识的目标图像,从而进行后续处理和应用。

当前的多数研究都依赖于比较光滑的反射面,对场景和目标要求都比较高。
而基于漫反射的研究则存在精度和实用性之间难以平衡的问题,停留在更换发射器和接收器,对成像原理的验证上,目前还没有能用于实际应用的系统和算法,也没有系统的成像理论把这些实验统一起来。

1.2  国内外研究现状及发展趋势
人类的观察范围会受到遮挡物的限制,与之相似,传统的成像技术也无法突破这种限制,因此非视域成像技术便应运而生,并且得到了越来越多的跟进。
非视域成像方法既可以对目标进行成像提供目标信息,又可使操作人员在距离障碍物较远的地方对目标进行观察从而保障了工作人员的安全。

1.2.1  激光选通非视域成像
在非视域成像中,比较简单的情况是反射面具有类似镜面的成像特性(比如玻璃、光滑的瓷砖等),因此可以进行主动照明并直接从反射面采集图像进行复原、增强等处理,从而对非视域场景中的目标有所了解。
激光发射到反射面之后将经历不同路径,被吸收,直接反射到接收器,经过反射打到目标场景。
打到目标的光会有一部分经过反射面成像到接收系统。
在接收端,通过选通滤除反射面的一次回波,可以接收到目标比较模糊的目标物体成像。
经过多幅图像的处理,就可以得到比较清晰的目标像。
这种非视域成像技术的原理如图1.2所示,所要观察的目标显然不在观察位置的视域之内,采用传统的直视视觉的方法无法观察到目标。


 
图1.2  非视域选通成像原理框图
Figure 1.2 Time-gated NLOS imaging

基于这种方法,2006-2007年,德国光子学和模式识别研究中心RIOPR (Research Institute for Optronics and Pattern Recognition)和瑞典国防研究中心(FOI)[3]进行了基于选通方法的研究,选择车窗作为中介反射面,对街道拐角处的人、车牌和金属人进行了实验(如图1.3)。


 
图1.3  RIOPR与FOI成像效果[3]
Figure 1.3 Imaging result of RIOPR and FOI

2011年瑞典的Ove Steinvall[4]在该项研究中继续推进,研究了窗户玻璃的反射特性,对比了窗户不同的反射角度对反射成像的影响。
然后进行了车身反射成像的研究,通过车盖反射观察了相距1.5m的人,分析了车盖弯曲程度对成像结果的影响,提出了非视域成像在更多领域的应用:通过窗户观察屋内和机载平台应用。

激光非视域成像技术可以看做是主动成像技术的特殊应用,其过程为:观察者控制成像系统发出脉冲激光照射到反射墙面上特定的扫描点,脉冲激光经反射面反射后到达被遮挡的目标,携带目标信息的脉冲激光返回反射面,部分通过墙上的探测器接收区域进入探测器,由成像系统接收。
激光照射墙上下一个扫描点。

墙面作为中介反射面在非视域成像过程中起到了至关重要的作用,它使照明激光可以绕过遮蔽物得以在观察者和隐藏目标之间折返,从而收集待观察空间的回波,从而恢复出场景信息。

2011年,FOI计算了激光距离选通系统在非视域成像模式下的一阶近似激光能量衰减因子,进一步研究了非视域成像过程中中介面的能量衰减和成像影响 [1]。

2013年,圣路易法德研究院和Wisconsin–Madison合作,利用距离选通技术研究了光源对非视域成像的影响并针对性地提出了帧对帧的反演算法[5][6][7]。

国内的研究成果主要出自北京理工大学和哈尔滨工业大学,北理工进行了基于选通成像的非视域研究,哈工大靳辰飞老师团队进行了多种尝试,在理论和实验研究中都有所建树,另外,清华、海军工程大学的学者也进行了相关尝试。

2011年许凯达等人采用激光距离选通系统进行了实验,开启了国内研究的先河[8]。
实验中采用激光波长 532 nm、脉宽 20 ns、频率 20 Hz[9]。
以常见材质如瓷砖、玻璃等为中介面,进行了成像研究。
实验结果表明干净玻璃的非视域图像质量要优于落灰玻璃图像,瓷砖作为中介反射面,成像效果较差。
研究中提出了距离选通非视域成像的图像对比度模型[10],分析了成像过程中表面特性和回波对成像质量的影响[9][11][12][13]。

2015年,韩宏伟完成了采用落灰玻璃进行反射的距离选通成像实验[14],哈工大硕士研究生翟建华作了关于精度对非视域成像影响的分析[2]。
 
 
图1.4  韩宏伟实验结果图[14]
Figure 1.4 Experiment results of Han Hongwei

2017年,哈工大在该领域的研究进一步深入,谢佳衡在先前研究的基础上,对椭球投射方法进行了分析,提出了球投射恢复算法,并研究了双中介面成像的情况,引入mean shift 滤波方法改善了成像弯曲畸变的问题[15]。

目前国内的研究都在很大程度上依赖于反射面,成像效果也不够理想,并没有能应用于实际场景的实验成果。


1.2.2  激光瞬态非视域成像
2007年起,美国MIT Media Lab研究采用飞秒激光和条纹管的实验组合进行了一系列研究[4][16][17][18],完成了隐藏场景的三维重建。
过程如图1.5。


 
(a)实验场景图 (b)实验所得条纹图 (c)重建图
图1.5  MIT研究原理图[4]
Figure 1.5 Schematic diagram of MIT

该实验首次使非视域成像不再受反射面的影响和限制,得到了广泛的追随和跟进。
这也是本次研究中要主要讨论的方法。
将其成像过程进一步梳理为图1.6。
2012年,麻省理工媒体实验室改进了反演算法[19] [20];

2014年,英属哥伦比亚大学Felix Heide等人利用二次漫反射光实现了对非视域场景的反演成像,其分辨率达到厘米量级,宣称其方法对环境光具有鲁棒性,并且可以用于房屋大小空间内的目标探测[21];
2015年,Andreas Velten等人将光子计数引入研究[22][23][24]。
研究中对比了ICCD和SPAD探测器,验证了探测器对非视域成像结果的影响,指出了中介反射面上激光反射和接收位置对成像的影响[2]。

 
图1.6  基于飞行时间的非视域成像过程框图
Figure 1.6 Schematic diagram of NLOS imaging using TOF

2016年,清华大学的Jingyu Lin等人提出非视域成像可以采用多频的ToF相机,提出了非视域成像的频域处理方法,减小了算法复杂度[25]。

1.2.3  其他方法及扩展应用
2011年,Rohit Pandharkar等人首次在非视域场景中引入移动物体,研究了杂乱环境下非视域目标的运动轨迹[2][26]。

2015年,哈工大靳辰飞团队利用光的直线传播原理对隐藏目标进行探测,通过三次激光散射,采用单光子APD进行探测,设置了特定的场景对非视域目标进行了三维成像[27]。


 
图1.7  靳辰飞团队实验原理图[27]
Figure 1.7 Experiment settings of Jin Chenfei
2017年,赫瑞瓦特大学进行了非视域目标的跟踪[28]。
2018年,波恩大学与圣路易法德研究院也进行了跟踪研究[29]。
同年,斯坦福大学的Matthew O’toole等人采用光锥变换和共焦成像的方法进行了成像研究,得到了目前最好的成像结果[30]。


 
图1.8  基于光锥变换的共焦成像方法装置图[30]
Figure 1.8 A confocal imaging method based on light cone transform

 
图1.9  基于光锥变换的共焦成像方法结果图[30]
Figure 1.9 Result of confocal imaging method based on light cone transformation

1.3  论文主要研究内容及章节安排
本文所述的激光非视域成像技术是指采用超短脉冲激光器和超快探测器配合,经过中介反射面进行反演成像并进行三维重建的技术方案。

本文首先将介绍国内外采用各种技术方案进行非视域成像的发展概况;
其次对非视域成像的理论和涉及的技术基础进行介绍;
接着对激光非视域成像过程中的光线传播理论进行研究,采集仿真数据;
然后对反演算法进行研究;
用多种方法进行点云重建;
最终结合仿真数据进行反演重建和总结。
 
论文的主要章节安排如下:
第1章 绪论。
介绍激光非视域成像技术的背景,明确研究意义。
概述目前研究进展,综述激光非视域成像技术的国内外进展动态。

第2章 激光非视域成像相关理论与技术基础。
首先将介绍本文所激光非视域成像技术的详细原理,再介绍本次仿真研究中所需要的三维模型读取、能量衰减、光线跟踪、反演算法和三维重建的基础知识,并介绍分辨率分析相关内容。

第3章 仿真建模。
将介绍本次研究中所设置的场景尺度及各部分物体的形状、器件参数及其模型,模拟数据采集过程,获取仿真数据。
 
第4章 反演算法研究。
首先将进行算法的可行性论证,其次对第三章中采集的模拟数据进行反演,并根据强度值对所得的点云进行筛选,呈现目标物体轮廓。

第5章 三维重建。
用标准点云数据进行三维重建,采用泊松、贪婪三角形和crust三种方法,对比重建结果,分析几种方法的优缺点和影响因素,确定本次研究中比较适合的重建方法。

第6章 总结与展望。
对第四章中得到的点云数据进行重建和分析,总结课题中所完成的工作和已取得的成果,同时展望未来研究值得尝试的方向。


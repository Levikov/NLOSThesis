\chapter{总结与展望}\label{chap:Conclusion}

激光非视域成像用于对隐藏目标进行成像,可以广泛应用于反恐、救援、侦查等特殊场合,而由于成像条件苛刻,相关研究推进缓慢。

本文进行了激光非视域成像的研究,回顾了非视域成像国内外研究成果与进展,从MIT初次提出实验方案到斯坦福大学的最新成果进行了梳理,明确了研究的瓶颈和方向。
以真实器件为基础,建立模型对数据采集过程进行了仿真模拟,并针对仿真数据完成了后期的反演重建。
研究中分析了距离、分辨率等多种因素对成像的影响,可以为实际系统的搭建提供可靠参考。
主要研究内容有以下几点:
1、研究了非视域成像相关理论,介绍了成像原理,计算了成像过程中的能量衰减,阐述了后期反演重建算法理论。
针对本文进行的仿真过程,增加了三维模型和光线跟踪方法的理论介绍。
详细分析了非视域成像的分辨率,指出了分辨成像的优势。

2、设定非视域场景,建立了器件模型和仿真过程中的响应测量模型,对数据采集过程进行了仿真模拟。
数据模拟采集过程中,首先采用基于点云的光线跟踪方法确定了目标的可见区域,然后对该区域的回波能量分布进行了计算采集,模拟实际系统,对能量进行采样量化之后存储,以备反演重建。

3、采用经典椭球重建算法,以模拟采集的数据为基础,进行目标所在空间的反演。
通过网格划分、数据匹配和点云筛选,初步得到目标所在的区域网格。
其中网格划分和点云筛选采用动态方法,减小了运算量。
得到了比较明显的目标轮廓。
对多种参数和场景进行了尝试,发现该种成像方法的成像场景非常有限,会受到多种因素的影响。

4、研究了多种表面重建算法,采用相同点云进行了对比实验。
其中泊松重建在点云密集、法向量精确的情况下可以得到接近原始表面的结果。
而当法向量缺失,点云分布不均匀时,基于网格的方法具有更明显的优势,尤其是powercrust方法。

研究结果表明,采用时间分辨的方法观察非视域目标的方法是可行的,但是成像方法对系统要求很高,反演算法也存在一定的缺陷和不足,导致成像系统非常脆弱,并不能满足现实应用的需要。
对于本文研究中存在的不足,可以做以下改进:
1、本文只做了仿真研究,忽略了一些实际问题,因而只能给实用研究提供一定的参考。
之后的理论研究中,可以在激光脉冲参数、光学系统特性、电路延迟等细节方面多做考虑。
搭建实际系统时,要注意噪声抑制、激光源和探测器之间的匹配,尽可能提高系统分辨率。

2、由于角度信息的缺失,目标空间各个点之间的数据会相互干扰,原本不存在目标的点会存在一定可能性,而原本的目标点反演出的强度值可能反而不太高。
虽然多次扫描可以在一定程度上减小这种干扰,但并不能完全消除,反演算法本身存在一定缺陷。
需要有其他算法或者理论知识做进一步补充。

3、该方法主要针对正对中介反射面的目标,对于倾斜物体或存在多个表面的三维复杂物体,一个角度的探测结果会由于各点之间的相互干扰而失去辨识度。
在之后的研究中可以考虑多角度探测,每个角度只保留离探测点比较近的目标点,再进行拼接。
这样可以在抑制目标点之间的干扰,提升成像质量。

4、论文研究过程,也是一个不断推翻、不断完善的过程。
在论文研究初期,计划要对点云进行表面重建,完成成像结果的可视化。
但是在实际研究过程中,发现三维目标的重建效果较差,点云筛选之后会完全失去辨识度,点云重建也失去了原本的意义。
从另一个角度来说,与照片类似,成像结果在视平面上的投影会保留一定的形状信息。
因此在现阶段的研究进度下,表面重建对非视域成像的意义并不是特别重大。


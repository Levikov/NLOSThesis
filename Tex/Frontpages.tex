%---------------------------------------------------------------------------%
%->> 封面信息及生成
%---------------------------------------------------------------------------%
%-
%-> 中文封面信息
%-
\confidential{}% 密级:只有涉密论文才填写
\schoollogo{scale=0.095}{ucas_logo}% 校徽
\title{激光非视域成像场景仿真与三维重建技术研究}% 论文中文题目
\author{镡京京}% 论文作者
\advisor{苏秀琴~研究员~西安光学精密机械研究所}% 指导教师:姓名 专业技术职务 工作单位
\advisorsec{}% 指导老师附加信息 或 第二指导老师信息
\degree{硕士}% 学位:学士、硕士、博士
\degreetype{工学}% 学位类别:理学、工学、工程、医学等
\major{信号与信息处理}% 二级学科专业名称
\institute{中国科学院西安光学精密机械研究所}% 院系名称
\chinesedate{2018~年~6~月}% 毕业日期:夏季为6月、冬季为12月
%-
%-> 英文封面信息
%-
\englishtitle{Scenario Simulation of NLOS Laser Imaging \\ and \\ Study on  3D Reconstruction}% 论文英文题目
\englishauthor{Tan Jingjing}% 论文作者
\englishadvisor{Supervisor: Professor Su Xiuqin}% 指导教师
\englishdegree{Master}% 学位:Bachelor, Master, Doctor。封面格式将根据英文学位名称自动切换,请确保拼写准确无误
\englishdegreetype{Engineering}% 学位类别:Philosophy, Natural Science, Engineering, Economics, Agriculture 等
\englishthesistype{thesis}% 论文类型: thesis, dissertation
\englishmajor{Signal and Information Processing}% 二级学科专业名称
\englishinstitute{Xi'an Institute of Optics \& Precision Mechanics, \\Chinese Academy of Sciences}% 院系名称
\englishdate{June, 2018}% 毕业日期:夏季为June、冬季为December
%-
%-> 生成封面
%-
\maketitle% 生成中文封面
\makeenglishtitle% 生成英文封面
%-
%-> 作者声明
%-
\makedeclaration% 生成声明页
%-
%-> 中文摘要
%-
\chapter*{摘\quad 要}\chaptermark{摘\quad 要}% 摘要标题
\setcounter{page}{1}% 开始页码
\pagenumbering{Roman}% 页码符号

在进行观察时,人眼和传统成像设备都会受到遮挡或阻碍,视域受到限制,存在一定的视觉盲区。
在救援、反恐、侦查等场合中,对非视域空间的观察至关重要,因此非视域成像技术受到了越来越多的关注。
随着探测技术的发展和算法研究的深入,非视域空间的成像成为可能,越来越多的学者进行了相关研究。

在对国内外研究现状进行深入了解的基础上,本文着重研究了非视域成像过程中的能量传输模型,针对实验场景选择器件并建模,对激光非视域成像过程进行了仿真模拟,并以仿真数据为基础,进行了后期的反演重建,分析了各种因素对成像的影响:

1)介绍了激光非视域成像仿真的相关理论,包括基于三维模型的光线跟踪、成像过程中的能量传输模型、采样量化、后期反演和表面重建理论。
在此基础上,设置了几个不同类型的场景,对数据采集过程进行了仿真模拟,采用仿真数据进行了反演重建,对非视域成像中的分辨率进行了详细分析,指出了分辨成像的优势。

2)设定多种目标,设置了有墙面遮挡的非视域场景,采用改进的点云跟踪方法追踪了目标的光线传播过程,对关键器件进行建模,模拟了数据采集过程,计算了场景的回波分布情况,采集了仿真数据。
仿真思路对系统研究起到了一定的参考作用。

3)采用经典椭球重建算法,以模拟采集的数据为基础,进行目标所在空间的反演。
通过网格划分、数据匹配、网格可能性赋值、筛选和轮廓提取,得到目标所在的区域网格,进而绘制出目标的成像图。
反演方法对场景要求比较苛刻,在一定情况下可以对二维目标进行成像,但是三维目标的成像结果失去了辨识度。

4)研究了泊松、贪婪三角形投影和powercrust三种表面重建算法,采用相同点云进行了对比实验,实现了不同方法的bunny重建,对比了几种方法的重建效果。

研究结果表明,采用时间分辨的方法观察非视域目标的方法是可行的,在一定场景下可以得到比较清晰的二维目标像,但是无法用于观察三维目标。
成像结果会受到诸多因素的影响,成像方法对系统要求很高,对场景和目标也有比较苛刻的要求,离实际应用还有很长一段距离。



\keywords{非视域成像,仿真,三维重建}% 中文关键词
%-
%-> 英文摘要
%-
\chapter*{Abstract}\chaptermark{Abstract}% 摘要标题

When observing, both human eyes and conventional imaging equipment can be obstructed by some shelters or obstacles, the viewing area is limited, and there is a certain visual blind area.
 In scenes such as rescue, counter-terrorism, and investigation, the observation of NLOS (non-line-of-sight) space is crucial, so NLOS imaging technology has received more and more attention.
 With the development of detection technology and the relative algorithm research, NLOS imaging becomes possible, and more and more scholars have conducted related research.

Based on the in-depth understanding of the research status at home and abroad, this paper focuses on the energy transfer model in the NLOS imaging process, selects and models devices for the experimental scene, and simulates the laser NLOS imaging process.
 Based on the simulation data, an inversion reconstruction was performed and the influence of various factors on the imaging was analyzed:

1) Introduce relevant theories of laser non-vision imaging simulation, including ray tracing based on three-dimensional model, energy transmission model during imaging process, sampling quantification, inversion algorithm, and surface reconstruction theory.
 Based on this, several different types of scenes were set up, and the data acquisition process was simulated.
 Inversion reconstruction was performed using simulation data.
 The resolution in non-vision imaging was analyzed in detail, and the advantages of resolution imaging was pointed out.

2) Set multiple objectives, set up NLOS scenes with wall obstructions, use the improved point cloud tracking method to track the targets’ light propagation process.
 Model key devices, simulate the data acquisition process, and calculate echo distribution of the scene, the simulation data was collected.
 The idea of simulation can play a certain role in actual system setup.

3) The classical ellipsoid reconstruction algorithm is used to invert the target space based on the simulated data.
 Through the grid division, data matching, grid probability assignment, screening and contour extraction, the target area is obtained, and then the target imaging map is drawn.
 The inversion method is more demanding on the scene and can image two-dimensional objects under certain conditions, but the imaging results of three-dimensional objects are not recognizable.

4) Three kinds of surface reconstruction algorithms, Poisson, greedy triangular projection and powercrust, were studied.
 The same point cloud was used for comparison experiments.
 Reconstruction of a bunny using different methods was achieved, and the reconstruction results of several methods were compared.

The research results show that it is feasible to use time-resolved methods to observe NLOS objects.
 In certain scenes, 2D target’s images can be obtained, but this method cannot be used to observe 3Dobjects.
 The results of imaging can be affected by many factors.
 The imaging method has high requirements on the system, and has more stringent requirements on the scene and the target.
 There is still a long distance away from actual application.



\englishkeywords{NLOS imaging, Simulation, 3D reconstruction}% 英文关键词
%---------------------------------------------------------------------------%

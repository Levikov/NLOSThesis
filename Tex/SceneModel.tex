\chapter{激光非视域成像场景仿真建模}\label{chap:Simulation}

仿真建模过程完成器件和场景模型搭建过程。
将本章内容分为场景设置,器件建模及测量模型、系统设置和数据采集模拟四节。
分别描述场景中各关键部分的分布,分析器件工作原理并搭建仿真模型,系统参数选择与设置,并完成数据采集的模拟过程。

将目标和反射面都离散化为小面片,反射特性满足理想漫反射,把光的传播过程简化为点与点之间距离和飞行时间的对应。
接收机记录回波强度随时间的分布。

3.1  目标文件读取与设定
进行仿真建模,首先要做的便是目标设定。
目标由模型文件读取,可用的模型分为两种,点云模型或者三维表面模型。
点云模型中需要存储物体各点的三维坐标、各点的法向量,对于表面材质不均匀的目标,还需要存储各点的反射率。
三维表面模型中还有拓扑连接关系。

本文中用到的模型为ply(Polygon File Format,多边形档案)文件,又名斯坦福三角形档案(Stanford Triangle Format),由斯坦福大学在The Digital Michelangelo Project计划中提出。

Ply格式常用来存储立体扫描结果,依次存储顶点信息和多边形连接顺序。
其中顶点信息除了三维坐标和法向量之外,还可以包括色彩、透明度等信息。
文档可以用ASCII或者binary格式存储。
文档以ply开头,然后是版本信息,之后是元素和属性。

对于本文实验来说,比较重要的元素关键字‘vertex’表示顶点,属性x.y.z表示三维坐标,nx,ny.nz表示法向量。
元素关键字‘vertex’表示顶点,属性x.y.z表示三维坐标,nx,ny.nz表示法向量,confidence表示该点存在的可能性,intensity表示反射强度。
关键字‘face’表示面,一般为三角形或者四边形。

在设定目标时,读取ply文件并提取出顶点和面片信息,经过一定的平移缩放,置于场景中一定位置,作为待成像的目标。


3.2  非视域场景设置
激光非视域成像的过程,可以看作是观察点从外围到中介面的转换。
只有目标在中介反射面上的投影面积足够大时,目标才具有辨识度。
因此在本次研究过程中,优先将目标设置为正对墙面放置。

如图3.1,我们将非视域场景设置为墙面遮挡,目标在墙角处,从(a)图可以看出,目标所在的位置不在观察位置的可见范围。
但是通过激光器和中介墙面的反射,接收机可以收到来自目标的三次回波数据。
目标的原始形状如(b)图所示。


   
(a)场景视图1                       (b)场景视图2
图3.1  场景简单光路图
Figure 3.1 Diagram of light path in the scenes

由简单到复杂,设置几个不同场景,模拟数据采集过程,进行反演获取目标轮廓。

假设目标材质均一,所有目标都以点云形式构造给出,点云文件中只包含目标各点的三维坐标和法向量。
以平面目标为主要研究对象,设置简单平面,图像组合和复杂人形三种,另外加入三维图形进行研究尝试。

3.3  激光非视域成像系统设置及器件建模
成像装置包括准直脉冲激光发射器,扫描镜和光电探测器及后续的信号处理模块。
非视域场景包括反射特性未知的遮挡物,此时假设非视域空间的位置和尺寸是已知或者可以估计的。
事实上,在大多数的实际场景中,这个假设是可以成立的。
场景中还需要一面可以充当中介反射面的墙壁,假设其位置和反射特性已知或可以通过传统方法测得。
这些先验知识保证了隐藏空间的反射特性估计是一个线性问题。

本课题中,最关键的几个参数是:激光器脉宽、重频、单脉冲能量,探测器时间分辨率和场景的空间分布。

激光器参数参照CALMAR公司的飞秒激光器,中心波长1550nm,脉宽100fs,重频50MHz,平均功率50mW。
脉冲在中介反射面上的扫描位置构成一个方阵。

探测器可以选用条纹相机,APD、PMD器件等等,本仿真系统系统采用ALPHALAS公司的ultrafast UDP-70-UVIR-P。
该探测器的响应时间可以控制在70ps以下,探测带宽可达5GHz,可以响应的波长范围为350-1700nm,对1550nm激光的探测效率大约在80%。

系统中的主要器件包括激光发射器、探测器和采样量化器件,另外需要建模的还有漫反射面和目标物体。

信号经过探测器之后,还需要经过采样量化,才能被存储传输进而用于反演计算。
采准直激光器逐点扫描中介面上的特定基准点{{P}}_{i=1}^N。
最常见的中介反射面——粉刷过的白墙,可以作为理想朗伯体,在周围空间中各个方向都有反射,因此可以断定,对于每次照射,都有一部分光线可以照射到目标场景。
因此,我们可以进行光线建模,从而将发射机和墙面抽象成在墙面的N个独立光源,逐次照射互不干扰。

与之类似,接收器通过一定的光学系统,接收墙面上特定点Q点处1cm的回波信息。
隐藏物体表面反射出的后向反射光就会部分入射到中介反射墙面上,经过中介反射面的朗伯反射进入光学系统,被探测器接收、采样并送到后续处理单元。
探测器、光学系统可以抽象为在墙上接收点处、具有一定衰减的接收装置。

参考校准
由于本项研究主要采集回波强度随时间的变化信息,因而回波被探测到的时间比信噪比更加重要。
在成像过程中,激光脉冲的稳定性、光学系统的光学特性、环境噪声、探测器引入的噪声和时延等都会影响激光的传输并造成结果的偏差。
因此需要采取一定手段来减弱这些因素造成的影响。

在回波信号中,有一部分是被中介面反射之后直接进入探测器的。
这部分信号可以作为参考样本。
由于中介反射面的回波信号可能被展宽,影响后续判断。
这里也可以用激光测距仪辅助研究。
目前市售的激光测距仪精度可以达到mm级别,对本项研究可以起到足够的辅助作用。

至于信噪比,可以通过暗环境操作、适当提升脉冲能量、控制重频、选择响应带宽比较窄的探测器等方法加以控制。

3.4  激光非视域成像数据采集模拟
3.4.1 基于点云的光线跟踪
相交判断依据:光线到面片重心的距离足够小。

遮挡判断:存在其他面片,光线到面片的距离更小。

具体判断方法:对于目标的每个面,计算其重心,计算光源到该点的距离。
寻找其他与该条光线相交的面并计算距离。
如果有更近的面,则初始面被遮挡不可见。
否则,保存该面及其距离,并进行后续反射的计算。

实现算法详细步骤
输入:原始的点云模型、平衡二叉树、栅格、最大的搜索半径 R、最近的离散点个数 N、阈值α 和 β 输出:布尔变量 intersect,交点的坐标和法向量 
步骤 1   计算光线与点云包围盒的交点,若有交点,即光线迭代点 A0,转到步骤 2;否则,返回 intersect = FALSE。
 
步骤 2   计算光线迭代点 Ai所在的栅格,若栅格在点云包围盒外,返回 intersect = FALSE;否则,判断当前栅格是否有离散点,若无,计算光线与邻近栅格的交点作为下一个迭代点 Ai+1,转到步骤 2,若有,转到步骤 3。
 
步骤 3   对于点云模型的所有离散点,利用平衡二叉树在最大搜索半径 R 的范围内搜索离迭代点 Ai最近的 N 个离散点,得到实际的 N′个最近的离散点(见上面的 Locate Points(p)算法)。
 
步骤 4   若 N′ = 0,则 Ai以步长 R 沿光线方向计算下一个迭代点 Ai+1,转到步骤 2;否则,转到步骤 5。
 
步骤 5   若 N′ < 3,则计算 Ai到 N′个离散点中最近点的距离,取其一半作为步长,沿光线方向计算下一个迭代点 Ai+1,转到步骤 2;否则,转到步骤 6。
 
步骤 6   若迭代点 Ai不在 N′个离散点的包围球内,则计算 Ai到其中最近点的距离,取其一半作为步长,沿光线方向计算下一个迭代点 Ai+1,转到步骤 2;否则,转到步骤 7。
 
步骤 7   由式(1)和式(2)计算过包围球圆心的局部平面的法向量 n,并计算光线与局部平面的交点 B。
若满足条件式(3),则得到相应的交点即为 B,其法向量即为 n,返回intersect = TRUE;否则,以迭代点 Ai到局部平面的垂直距离 h 的一半作为步长,沿光线方向计算下一个迭代点 Ai+1,转到步骤 2。
 
上述算法通过改变光线跟踪的参数(最近离散点的数目),即可渐进地多分辨率显示原始的点云模型。
对于噪声多的原始的点模型,设置较大的最近离散点的数目,以有效地减少其绘制的噪声;对于噪声少的原始的点模型,设置较小的最近离散点的数目,以更多地显示其局部几何特征。
以三维图bunny为例,基于点云的光线跟踪结果如图3.3。
面片三维模型的跟踪结果如图3.4。


   
(a)正对观察点方向                (b)观察点垂直方向
图3.3  基于点云的光线跟踪结果
Figure 3.3 Results of ray trace based on point cloud

 





从图中可以看出,经过光线跟踪方法的处理,三维模型中正对观察点的部分被完整保留,而背对观察点的绝大部分可以被剔除。
由于三维模型的面片较大,用重心表征面片位置不够准确,背向点的剔除有所疏漏,但该方法基本实现了光线跟踪,比较完整地提取出了特定角度下目标的可见部分。


3.4.2 多次漫反射回波强度计算
Equation Section (Next)根据2.3中所述,对于每个发射-接收对,从目标接收到的能量表达式为
	  	 (3.1)

该能量被光电探测器以一定效率转换为电信号。
光电转化的过程中,如果将探测器正向(反向)阶跃响应的光电流上升(下降)到终值(初值)0.63(0.37)倍所需要的时间定义为其响应时间 ,则其带宽 和响应时间 之间符合下面的关系:

	  	(3.2)

这表明,输入探测器的光功率与其输出的光电流之间存在一阶惯性传递函数关系:

	  	(3.3)

探测器的噪声主要包括散粒噪声 和热噪声 ,二者都是白噪声[46],其均方值分别是

	  	(3.4)
	  	(3.5)

3.4.3 时间强度信号的采样量化
后续反演过程主要依靠时间-强度信号,因而需要将该信号进行数字化处理并存储。
在经过探测器之后,由TCSPC对该模拟信号进行采样和量化,以直方图形式输出并存储。

采样量化参数 为准,该TCSPC的抖动控制在10ps以内,计数栅格可达1ps,也就是距离分辨率0.3mm的技术水平,可以很好地解决激光非视域重建过程中的时间分辨问题。

采样频率必须和发射脉冲的参数相匹配,同时也将直接决定反演算法的精度。

量化采用单光子能量为度量标准,增加结果的可读性,强度值将以该时刻回波光子数表达。

 ,其中h为普朗克常量,ν是波长,取1550nm,N即表示该时刻被计数的光子数。


3.4.4 模拟采集流程
流程如图3.5所示。
激光器对散射墙面上一定区域的多个点(1m*1m,10*10个点)进行扫描,每次扫描都计算光源-墙面上激光照射点-目标-有效接受区域中心-接收到的距离,根据激光反射衰减方程计算其对应功率。
将得到的这些精确数据整合并存档,作为该场景条件下的精确数据。

问题:平整表面的三维模型中,三角块会很大,影响判断。
因此,是否遮挡的判断,不能采用单一的距离阈值作为准则,而要动态适应。
甚至可以考虑将目标进行三维重建之后再输入场景进行仿真。
但是,经过重建的目标并不能完全还原目标本来的形状。


 
图3.5  数据采集过程框图
Figure 3.5 Block diagram of data acquisition process.

3.5  回波信号时域空域分布
对于一个接收位置,经过数据采集过程的仿真,可以得到回波信号的时域分布。
此外,在中介反射面上不同位置,回波分布也会大不相同,因而如果对墙面多点进行探测,同时可以得到回波在该面上的空间分布。


 
图3.6  光子数-时间仓直方图
Figure 3.6 Photon-time bin histogram

 
图3.7  中介反射面回波强度分布图(归一化)
Figure 3.7 Space distribution of echo on the wall

回波在中介反射面上的分布与目标位置有很大关系,基本上是目标的投影位置。
因而在成像过程中,中介反射面上的接收位置将直接决定回波信号的强弱,进而影响反演成像结果。
接收器要尽可能正对目标投影,才能保障回波的强度,进而保证有足够的能量用于反演重建。
在实际应用中,目标位置是完全未知的,因此需要进行预估或者多次探测取较好结果。

3.6  本章小结
本章详细介绍了仿真建模过程,包括目标设定、场景设置、器件建模、系统设置和数据采集过程的模拟,明确了场景中的目标形状尺寸、空间分布情况和各表面的反射特性,阐释了器件的工作原理和信号转换过程,得到了采集数据的仿真结果,用于下一步成像,并观察了回波在中介反射面上的分布,为系统位置设定提供了参考。


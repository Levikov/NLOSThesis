\chapter{非视域成像理论与技术基础}\label{chap:Basis}

激光非视域成像原理
非视域成像建立在主动成像技术的基础上,辅以高精度瞬时数据采集方法,可以收集到来自目标的精确信息。
其基本的成像原理与激光主动成像相同,都是发射激光照亮目标并采集处理回波信号来还原物体形状的方法。
但是由于非视域成像会经过三次漫反射,无法直接得到目标图像,只能得到回波信号的时间信息来反演推断,因此对探测器的时间分辨率要求更高,无论是采用条纹相机的经典方法,还是采用APD的尝试研究,都在一定程度上借助了瞬时成像技术。

2.1.1  激光主动成像
根据是否对目标及背景进行人工照射,成像方式被分为主动和被动成像[31]。
被动成像依赖于目标和背景自身发光或反射其他光源的光,在光线比较均匀且足够强烈时可以获取目标得细节信息,功耗相对较小,系统简单,可以隐蔽;但受环境约束,当光线较弱时无法得到高质量的成像结果。
主动成像则需要主动照射目标,借助反射光对目标及背景照射成像,能够主动增强目标信号,对目标进行成像,一定程度上摆脱了对环境的约束,但是耗能高,在观察目标的同时会暴露系统自身位置[32]。
 
激光主动探测成像过程中,采用扫描镜可以移动光线照射位置,照亮目标的特定区域位置,选择性地对目标进行成像,配合其他技术,可以实现识别、跟踪等等。
主动探测中经常采用激光源,激光具有能量集中、方向准确等特点,可以进行定点照射。
照射待观察物体,提高目标反射信号的能量,通过收集目标景物反射回来的光信号,可以提高目标图像的对比度,可以适应比较差的周边环境[33]。
其原理如图2.1所示。

 
图2.1  激光主动成像原理框图[33]
Figure 2.1 Block diagram of laser active imaging

成像过程中,主动照射目标所在空间,接收器收集反射信号,经相机获得图像信息并进行后续处理[34]。
依据激光照射区域,可以分点光源照明和全局照明两种。
点照明的方式每次都只照亮目标上的一个小区域,回波可以精确定位,再通过拼接和融合拼出完整的目标表面,这种方法定位精准,但是需要多次照射,效率比较低;全局照明的方法效率较高,但是各个点的回波会相互干扰,得到的成像结果会出现模糊、畸变等问题。

2.1.2  瞬时成像
在传统成像中,由于光速很快,在一定尺度的成像中,光到达物体表面的时间被看作是相同的,成像被看作是一个全局同时的稳态过程。
视频可以被解释为不同但静态世界的图像序列,因为每个帧的曝光时间足够长。
与光的瞬态特性的时间尺度相比,它们的样品光非常慢。
在一个房间大小的环境中,一个微秒曝光(集成)时间足够长,可以使光脉冲完全遍历所有可能的多路径,这是由于场景元素之间的相互反射,并达到稳定状态。

瞬时成像则定义了包括飞行时间信息、也就是时间分辨的光传输模式。
在这种模式下,光的速度被看作有限的,当光在一个场景中散射时,它需要不同的路径,而较长的路径需要更长的时间来穿越。
即使是单一的光脉冲也能在时间上演化成一个复杂的模式。
光的瞬态传输理论描述了光线与场景的相互作用。
入射照明提供了第一组光线,可以在场景和照相机的其他元素上传播。
直接反射之后是一种复杂的模式,它的动力学是由场景元素的场景几何和材料属性决定的。
光在不同的路径上传播,将花费不同的时间,光到达接收点的时间和强度随着不同的距离而变化。

我们认为一个场景S由M个小平面切面(单位面积的斑块)组成,记为 。
场景中的几何数据 ,其中各切面的位置 ;距离矩阵 ,  是面  和 之间的欧氏距离;法线矩阵 由每个单位表面 的法向量 组成;可见矩阵 ,其中  表示面  被 遮挡。
为了便于分析,我们将摄像机(观察者)和照明源看作是由 表示的单个切面。

光以有限的延迟时间在场景中穿越距离  是在场景之间穿梭的光线的集合,t表示离散的瞬时时间,。
瞬态光传输由以下动力学方程控制,我们称之为瞬态光传输方程:Equation Section (Next)

	  	(2.1)

方程2.1说明t时刻 到的 辐射总量 是由直接发射辐射 和经过其他路径间接辐射加权延迟的总和。
为了简单起见,设光速c = 1,则传播延迟 等于距离 。
我们假设所有的延迟 都是一个单位延迟的整数倍。
标量权重 表示光从  经 到 的衰减
 
	  	(2.2)

其中  是反射率,取决于材料的反射特性,遵循反射定律 , 是入射角,  是出射角。
另外,面片自身不会存在交互作用,当  或  时, 。
假设场景为静态,材料属性在成像间隔上是恒定的。
代表光源和探测器的 不参与内部反射,即 
我们使用发射器面片 来模拟照明,场景中的所有其他切面都是非发射的,  。
面向对象。
光照是辐射 的集合,代表在不同的时间瞬间,发射到所有场景的光线。
面 的输出光 。
对于整个场景,光线传输向量 ,其中包含了  个标量辐照度。
我们只能观察到 指向相机 的投影。
在每个时刻t,都需要记录M个强度值的一个矢量 。

以上分析可以推广到包括多个来源和观察者在场景中的任意位置,最经典的应用便是条纹相机成像。
条纹相机将输入光的时间轴转化为空间轴成像在荧光屏上,精确记录了每个时刻接收到信号的强度,分辨率可以达到亚ps级别。
其成像模型包括广义传感器和脉冲光源。
每个传感器像素在场景中观察一个独特的切面,传感器位置的像素 捕捉并记录到 处辐射照度随时间的变化,根据采样传入的辐照度,创建一个3D时间图像, 。
传感器采集与照明光源是同步的,精确测量到达时间与发射时间的差异(TDOA)。

瞬时成像目前主要应用于生物医学研究、雷达成像和地质地理研究,以其超高精度突破了传统成像方法的限制。

使用超短脉冲定向照射,搭配超快探测器接收回波,可以进行瞬间成像,形成一组切片图像;或者精确收集每个瞬间的回波信息,进而推演空间中物体的分布。
可以扩展的主要应用场合是散射介质中物体的成像和被拐角遮挡物体的恢复。

2.2  多次漫反射能量衰减模型
在瞬时成像的框架下,场景的能量衰减模型将具体到点。
根据日常场景,假设中介反射墙面和目标物体均为朗伯体,满足理想漫反射条件。
相对于反射衰减的能量,激光在短距离上的衰减可以忽略不计。

如图2.2,脉冲激光器准直输出,打到反射墙面上的P点。
接收器的有效接收面积很小,只能收到Q点周围很小区域的回波。
其中P点反射出的一部分漫反射光照射到目标M,经过目标二次漫反射到Q点。
光的整体传播路径为激光器——P——M——Q——接收器。

 
图2.2  场景简化图
Figure 2.2 Simplified scenario diagram

为了方便计算,先用方波脉冲代替实际中的高斯脉冲,设定激光器峰值功率为$P_0$,光束直径为$A_0$,漫反射面的反射率为$\rho$,目标表面反射率为$\rho^\prime$,接收器探测效率为$\tau$。

先不考虑激光在空气中的衰减,将中介反射面和目标看作理想漫反射体,则其辐射亮度L与观察角无关,是一个定值。
光在中介反射面的传播将符合朗伯余弦定律

	  	(2.3)

假设小面元上的功率是均匀的,即小面元上功率密度处处相等,则小面元向半球空间的辐射功率

	  	(2.4)

因此
	  	(2.5)

激光束横截面半径$\omega$与该横截面距激光束腰的距离R和激光束散角$\theta_0$的关系为

	  	(2.6)

仿真中脉冲激光器采用光纤准直输出,在实验距离比较小的情况下可以不考虑光束发散,也不需要考虑激光在空气中的衰减,我们可以假设P处光斑的面积和出射激光相同

	  	(2.7)

P处小光斑的接受功率

	  	(2.8)

辐射亮度

	   	(2.9)

设目标的有效接收面积为$A_i$,对应$P$点的立体角$\Omega_1$很小。
再假设P处面元上的功率密度处处相等,则目标的接收功率为

	  	(2.10)

式中$\theta_{11}$是指反射墙面法线与$\vec{R_1}$的夹角,$\theta_{12}$是指目标法线与$\vec{R_1}$的夹角。

设墙面法线为 。
探测光学系统轴线与 的夹角记作$\theta_{31}$,探测器对应于反射面的有效面积为

	  	(2.11)

实际上,接收机通常是正对墙面的,$\theta_{31}=0$。
墙面上的该面积区域接收到来自目标的辐射为

	  	(2.12)

式中$\theta_{21}$是指目标法线与$\vec{R_2}$的夹角,$\theta_{22}$是指反射墙面法线与$\vec{R_2}$的夹角。

接收机接收到的光功率

	  	(2.13)

式中$\theta_{32}$是指接收机中心轴线与$\vec{R_3}$的夹角,有$\theta_{32}=0$。

联立以上各式,

	  	(2.14)

如果目标表面不规则或者表面较大,需要对目标进行积分求面积

	  	(2.15)

再考虑到瞬时成像中必须考虑光的传播时间,时间t接收到的光功率
	  	(2.16)
2.3  基于点云的光线跟踪原理
光线跟踪法用来模拟光线的传输路径,判断光线与目标的交点。
对于可以用函数表达的规则曲面,可以用光线方程与之解交点。
而对于点云或者三角面片表示的三维模型,解交点的方法不再适用,此时需要采用改进方法。

本文借鉴了一种基于点云的光线跟踪方法[35],其基本步骤为:
1)将点云模型以二叉树结构存储,点云所在空间划分为网格形式。

2)设定光线迭代点,模拟光线在空间中的传播,直接穿过空网格。

3)对于一条特定光线。
搜索一定距离中最近的几个目标点,解出局部平面,判断光线是否能穿过该平面,求解交点;否则,跳2),计算下一个迭代点。


 
图2.3  基于点云的光线跟踪
Figure 2.3 Ray trace based on point cloud

本文以此为基础,对目标进行光线跟踪。
当从某一方向直接观察时,光路可逆,能与入射光相交的点一定能有反射光被探测到。
而在主动成像时,发射与接收不在同一位置,对入射光和出射光都要进行跟踪。

具体跟踪步骤为:
1)设定光源位置,将点云模型以二叉树结构存储,点云所在空间划分为网格形式。

2)设定光线迭代点,模拟光线在空间中的传播,直接穿过空网格。

3)对于一条特定光线。
搜索一定距离中最近的几个目标点,解出局部平面,判断光线是否能穿过该平面,求解交点;
4)将交点处的反射光离散处理,模拟光线传播,看能否直接到达接收位置;
5)如果能,计算能量衰减;否则,跳2),计算下一个迭代点。

当目标为点云文件时,采取上述方法。
当目标为三角面片构成的三维模型,将面片用具有一定体积的理想点替代,用类似方法完成三维模型的光线跟踪。

2.4  反演算法原理
每个数据对,其在隐藏空间中对应的目标位于一个椭球面上。
给上一步得到的精确数据加入一定噪声作为系统接收机接收到的数据,经过滤波-反演-渲染完成重建。
这里将其中的反演步骤作为重点。

2.4.1 经典椭球反演算法原理
经典重建算法理论基础:漫反射成像椭球重建机理 
 
图2.4  椭圆重建原理图
Figure 2.4 Ellipse reconstruction

以二维平面图为例,对于目标点A,当发射点(S1)和接收点(R)已知时,根据激光在其间经过的总距离d,可以知道目标点A在以S1和R为焦点,d为长轴的椭圆上。
将R固定,更换S点的位置,可以得到另一个椭圆。
如果数据足够精确,这两个椭圆的交点就是A点的准确位置。
与此类似,将平面扩展到空间,三个椭球面的交点就可以确定一个目标点。
这也是目前大多数研究采用的方法,具体反演计算过程如下:
(1)将待探测空间划分为网格 ,每个网格以其中心坐标表示,各网格的置信度置零;
(2)对于每一个收集到的光子数 ,计算出满足时间关系的体素集合,即为该组数据对应目标的可能位置。
给这些体素的置信度增加 ,其中 是SPAD区域的距离修正项, 是体素点的修正项, 是朗伯项。
 是r3和r4的夹角。

(3)滤波:进行拉普拉斯滤波并归一化。

(4)阈值筛选:如果一个网格的置信值和其周围一个小邻域的置信值都高于阈值,该体素被保留。
经过滤波之后的归一化处理,置信值被保留在了0~1之间。
阈值的选择需要根据具体情况而定。

2.4.2 光锥变换反演算法原理
由式 (2.16),以中介反射面为平面$z=0$,对于中介反射面上的扫描点  ,探测到待重建空间 的回波强度分布为
  	(2.17)
该分布在x和y方向具有平移不变性,其中 。
做几个代换,  
  
  	(2.18)
记 , ,  ,则有 。

其中h是一个平移不变的3D卷积核, 是能量随z轴的衰减与采样。
  是能量τ在时间轴上的衰减采样。
 和 都有闭合表达式。
式(2.18)被称为光锥变换(light-cone transform , LCT),可以被离散表达为 ,其中 是探测到能量的向量表达,  是隐藏表面反照率的向量表达。
矩阵 代表平移不变的3D卷积算子。
矩阵 和 分别表示时域和空域的变换操作。

	  	(2.19)
该变换的离散模拟由矩阵向量的乘积 给出。
对于其中的一次扫描,设  ,探测器可以收到来自目标区域 的信号,瞬态图像和相应的变换矩阵可以表示为
	  	(2.20)
其中  
离散图像信息可以表达为
	  	(2.21)
其中 是光传输矩阵,每个矩阵都独立应用于各自的维度,因此可以以高效的内存方式应用于大规模数据集。
矩阵 表示用4D超锥的移位不变3D卷积(即与离散化内核h的卷积),在变换域中对自由空间中的光传输进行建模。
这些矩阵一起表示离散的光锥变换。

离散光锥变换提供了快速且高效的存储方法来计算前向光传输 和反向光传输 ,而不用明确地形成任何矩阵。
 使用卷积定理,矩阵向量的乘法在傅里叶域中可以由元素乘法实现,可以在很大程度上提升计算效率。
类似地,矩阵向量与其逆的乘法计算可以采用傅立叶域中的元素分割。

反演方法
假设离散光传输模型的噪声模型满足
	  	(2.22)
其中 。
 是白噪声。

根据维纳去卷积滤波器,均方误差最小的解 可以表达为
	  	(2.23)
其中F是指3D离散傅里叶变换, 是一个包含移动不变3D卷积核傅里叶变换的对角矩阵。
 与频率有关,代表信噪比SNR。

具体到NLOS问题,非视域成像的反演过程可以表示为
	  	(2.24)
其中 是3D离散傅里叶变换, 是估计所得隐藏表面的反照率,   是包含3D卷积核傅里叶系数的对角矩阵,  代表成像过程中的信噪比。
基于维纳滤波,可以通过反复迭代求出 。


2.5  三维重建方法 
在逆向工程和激光雷达成像过程中,当得到点云的分布之后,需要采用表面重建的方法建立拓扑关系,得到平滑表面。

常用的三维重建算法中,应用最广泛的是泊松重建方法和基于网格划分的重建方法。
此外还有参数曲面,变形曲面和细分曲面等方法[36]。

泊松重建是最具代表性的隐式曲面重建方法[37],由Kazhdan M等人在2006年提出,立足于最简单的事实:物体表面将空间分内外两部分。
可以构造1,0的指示函数表示。
通过求解出这个函数然后进行等值面提取,从而得到物体表面。
而求解这个指数函数,就是构建一个泊松方程并对泊松方程进行求解的过程。

方法基于泊松方程。
重建中由梯度关系得到采样点和表面内外指示函数的积分关系,根据积分关系利用划分块的方法获得点集的向量场,计算指示函数梯度场的逼近,构成泊松方程。
根据泊松方程使用矩阵迭代求出近似解,提取等值面,对所测数据点集重构出被测物体的模型。
泊松方程在边界处的误差为零,因此得到的模型不存在假的表面框。

基于网格曲面的方法更符合实际工程中的多边形模型,网格也更便于数字计算和显示[38],更适用于散乱点云的重建。
常用算法包括零集法、α-shape算法、基于Voronoi图的算法和贪婪三角形投影重建算法[39]。

对比几种网格重建算法如下:
零集法通过拟合来重建曲面,允许原始点有噪声干扰[38]。
其效果处于网格和精确重建之间,虽然没有经过理论论证,但是有很好的实验结果,对于均匀密集的采样点,可以很好地重建。
但是算法依赖于采样点邻域其他点的情况,因此不能处理杂乱稀疏点云。

α-shape算法经过了理论论证,在一定条件下能无限接近原表面[40],但是需要点云均匀,密度已知;否则需要进行算法改进[41]。
当原物体分为几块时,可以根据密度进行剔除,去掉不同密度点之间的点集[42],以分离不同的物体。
类似地,α-shape可以用于处理切片数据[43]。
当点云密度未知时,α-shape可以先计算所有的α形状族,进而挑选适当的的α[44]。

Voronoi理论保证依据较为充分[41],可以用于重建不均匀的点云数据。
先进行Delaunay三角剖分,再选择合适的Delaunay面片,可以生成与原始曲面拓扑一致的网格曲面,其中需要进行筛选[45]。
这一步需要借助该处的法向信息,所以该方法不能用于边缘陡峭目标的重建。

贪婪算法的核心思路是通过局部最优达到全局最优。
贪婪法用于表面重建时,先把三维问题放在二维平面处理,再根据二维拓扑连接构建三维曲面。
投影时需要得知点云法向,因而要先估算点云的法向量。
然后将有向点云投影到局部坐标平面内,再在平面内进行网格三角化,根据平面内各点之间的拓扑关系获得一个三角网格曲面模型。

2.6  激光非视域成像分辨率分析
一般成像系统的分辨率主要包括时间、强度、空间和角度。
主动成像的分辨率由激光源、目标位置和接收器共同决定。
但是,在非视域成像过程中,光线要经过三次漫反射,这样的成像机制虽然可以突破性地扩展成像范围,但成像难度偏大,对系统和场景都有一定要求,同时也丢失了系统对角度的分辨,增加了成像过程中的不确定因素。

强度分辨率是指可以分辨的强度差,决定因素是探测器的响应和采样量化过程中的方法和参数选择。

空间分辨率是指最终可以区别的目标位置间隔。
对于直接成像,该分辨率与距离、透镜参数等因素直接相关。
但是由于非视域成像会经过多次反射,信号与目标点之间也没有明确的对应关系,因此无从分析。
在本项研究中,中介反射面上的探测器接收点是惟一的,激光扫描点和强度-时间直方图是可以一一对应的。
这种情况中,空间分辨率可以忽略。

激光非视域成像系统中,对成像最重要的是时间分辨率,即系统可以分辨的回波时间间隔,主要取决于传感器的分辨率、激光器的脉宽和重频。

2.6.1  传统成像局限性
即使使用传统相机,对于目标在z方向的微小位移,原则上也是可以捕捉的,但事实上,这是一个病态问题。
假设一个深度为 的近平面场景,由远处的光源均匀照明,目标反射的光线约为 ,其中I为总的入射强度。
将目标移动 ,强度变化 。
因而可以得到 。
在一般的场景中, ,可以得到 ,这对探测器提出了极高的要求,目前还没有探测器可以达到这个水平。
另外,非视域成像过程中要经过三次漫反射,信号会受到诸多噪声影响,这个微弱的强度变化会淹没在噪声中,进一步增大信号采集的难度。

2.6.2  分辨成像优势
高分辨率的瞬时成像技术可以有效规避这两个问题。
只要探测器分辨率足够高,满足  ,目标不同位置上的回波就会处于不同的时间栅格,在反演时就可以分辨。
此外,该栅格数据在空间中的反演结果将是一个椭圆环,如图2.5。
圆环半径 ,  ,圆环宽度 。
圆环面积 对于一个面积为 的目标小块,  。
经过三次漫反射,场景中 ,因此 。
因此要分辨这个变化,所需要分辨的强度增量与线性比例成正比,而不是先前的二次关系。
在一般场景中,该比例在10-2即可。
此外,激光源的分辨率要和探测器相匹配,甚至优于探测器,否则也达不到实用精度。


 
图2.5  分辨成像
Figure 2.5 Resolution imaging

在非视域成像过程中,对形状估计设置比较重要的参数有:传感器与激光器的时间分辨率,传感器的强度分辨率,光源功率与信噪比和目标物体的几何形状。
时间分辨率高的优点主要体现在:精确的时间关系和对背景噪声更好的容忍性。

2.6.3  分辨成像的实际应用
时间分辨率对激光非视域成像至关重要。
传统成像中,目标物体到达成像面的时间差被忽略,成像的清晰程度主要由透镜的空间分辨率决定。
而在非视域成像中,我们要通过回波信息来判断目标各点到探测器接收点之间的距离进而判断目标位置,因此时间分辨率至关重要。
目标上各个点的反射光到达中介面上的时间相差很小,以目标在米级大小为例,1m/c=3ns。
3ns的时间分辨率,能分辨的目标点为米级范围。
而对于一般物体,只有精确到厘米级才能分辨出其大概轮廓,对细节的观察对分辨率要求更高。
因此,系统的时间分辨率至少要在ps级别。

目前市售的时间相关单光子计数器(TCSPC)制造商中,上海昊量光电的产品达到了抖动10ps以内,计数栅格可达1ps,也就是距离分辨率0.3mm的技术水平,很好地解决了激光非视域重建过程中的时间分辨问题。


 
图2.6  xy平面分辨率计算
Figure 2.6 resolution of the plan xy calculating

以0.3mm分辨率进行分析,z轴的分辨率与该数值直接对应,在xy平面上的分辨率与目标距离墙面的距离有关。
本文中,为了模拟真实应用环境,最初把目标与墙面之间的距离在了4.5m,
  
解得如表2.1所示的可分辨数值

表2.1  4.5米处物体xy平面可分辨距离
Table 2.1 Distinguishable distance of object at z=4.5m
i	1	2	3	4	5	6	7	8	9
 
0.0520	0.0215	0.0165	0.0139	0.0123	0.0111	0.0102	0.0095	0.0089

系统在z方向的分辨率很高,在xy平面上的分辨率随着目标偏离中间的位置会逐渐变化,正对位置时分辨率只有0.052m,这样的分辨率在观察车辆、装备等大型物体还算勉强,但是对于日常的室内或者拐角场景,该分辨率不足以看清楚任何一个物体。

当物体与目标之间的距离是1m时,同样的方式可以计算得到表2.2,可以看到;距离为0.5m,xy平面如表2.3,这个分辨率基本上可以保证物体的可观测性。


表2.2  1米处物体xy平面可分辨距离
Table 2.2 Distinguishable distance of object at z=1m
i	1	2	3	4	5	6	7	8	9
 
0.0245	0.0101	0.0078	0.0066	0.0058	0.0052	0.0048	0.0045	0.0042

表2.3  0.5米处物体xy平面可分辨距离 
Table 2.1 Distinguishable distance of object at z=0.5m
i	1	2	3	4	5	6	7	8	9
 
0.0173                                  	0.0072	0.0055	0.0046	0.0041	0.0037	0.0034	0.0032	0.0030
2.7  本章小结
本章详细介绍了激光非视域成像技术的相关理论与技术基础,包括成像原理,数据模拟采集过程中的三维模型读取、能量衰减计算和光线跟踪,以及后期的反演算法和三位表面重建方法,并对系统的分辨率进行了分析,阐述了采用分辨成像的优势和必要性。

经过分析,系统对探测器的分辨率要求非常高,这也是目前系统发展的瓶颈之一。
非视域成像的分辨率由光源、探测器等多种因素决定,研究过程中要密切配合。


\chapter{激光非视域成像仿真数据的反演}\label{chap:Simulation}
采集到光子数随时间分布的直方图数据之后,利用一系列直方图,可以反演得出非视域空间中的反射情况,从而得知空间中目标的分布位置。
具体原理如2.4所述。
本章采用多个二维目标,进行非视域场景的反演及影响因素的分析。

\section{椭球累加反演}
如前所述,根据对隐藏空间和中介反射面的先验了解,场景反演过程可以看作一个线性反演过程。
每一个激光发射点-探测器对都会以直方图的形式输出一组时间-强度数据,只是在每个时刻的回波强度。
对于每个数据,都可以在空间中找到一系列与之对应的点,该点处存在目标的可能性与回波强度成正相关。

反演重建的关键步骤如图4.4。


 
反演重建流程图
Figure 4.4 Flow chart of inverse and reconstruction

该算法根据各个体素权值的大小进行筛选,权值代表各个体素是目标点的可能性。
对于每组距离数据,给处于其椭球面体素赋的权值I(k)会直接影响到最终体素的选取。

二维目标的反演
三维目标的反演
\section{光锥变换反演}
在4.1所示场景中,采用光锥变换的方法,将发射点和接收点放在同一位置,采集数据并进行反演,得到反演结果。

二维目标的反演
三维目标的反演
 
\section{结果对比与定量分析}
二维图像重建均方误差



三维成像可懂度(?)	

运算量分析
\section{本章小结}
本章实现了非视域空间的反演。
采用传统的椭球累加方法和光锥变换方法分别进行了非视域空间的反演重建,得到了非视域目标物体的大致轮廓。

椭球累加方法不适用于三维目标的成像,三维目标在z方向的分布会严重影响反演过程中各体素的强度取值。
这也解释了大多数文献都停留在二维目标成像的现状。
光锥变换是一种不断叠加逼近精确求解的方式,不受目标形状的限制。
 

\chapter{激光非视域成像仿真数据的反演}\label{chap:Simulation}
采集到光子数随时间分布的直方图数据之后,利用一系列直方图,可以反演得出非视域空间中的反射情况,从而得知空间中目标的分布位置。
具体原理如2.5所述。
本章采用多个二维目标,进行非视域场景的反演及影响因素的分析。

4.1  反演算法可行性论证
为了验证算法的可行性,先将目标设为一个很小的点进行仿真实验,场景坐标和范围如表4.1所示。
由图4.1中可以看出,除去因体素划分造成的系统误差,反演结果基本与目标点一致。


表4.1  场景坐标设置
Table 4.1 Location of the scenario
物体	位置
激光器	(4,3,6)
探测器	(4,3,6)
目标	(1,1,4.5)
探测器在中介反射面上的接收点	(4,3,0)
发射器在中介反射面上的扫描点	([3.5:0.1:3.7],
[2.5:0.1:2.7],
0)	([3.9:0.1:4.1],
[3.9:0.1:4.1],
0)	([3.5:0.1:4.5],
[2.5:0.1:3.5],
0)
反演结果	(0.994,1.008,4.5)	(0.982,1.026,4.5)	(1.002,0.996,4.5)

可以看到,反演结果与原位置有一定的偏差,偏差的大小与激光扫描位置有关。
增加扫描点可以有效减小偏差。
究其原因,在建模过程中,反演结果是几组数据的加权累计。
当扫描点比较少时,距离较近、权值较大的数据对扫描结果影响很大,反演误差较大。
当扫描点增加时,大量数据加入计算,权值的影响会相对减弱,结果会越来越准确。
但是,增加扫描点会增大数据量,影响计算效率。
因此在实际研究中,需要在效果和效率之间折中选取恰当的扫描点数。


   
图4.1  单点反演效果图
Figure 4.1 Inversion results of single point

接下来用两个目标点来进行实验,以(1,1,4.5)和(1.1,1.1,4.5)两点为目标,11*11个扫描点为例,可以看到两个点仍然可以被较为准确地反演出来。
图4.2 (a)是全局最大值所在点,图4.2(b)是局部最大值。
可以看到,当两个目标之间有一定距离,整体的距离值有一定差异时,全局最优不能作为选择标准。
因此在后续计算中,采用全局和局部最值加权筛选的方式进行。


  
图4.2  双点反演效果图
Figure 4.2 results of two points

缩小两点之间的横纵距离改至1cm和3mm,将目标点改为以(1,1,4.5)和(1.03,1.03,4.5); (1,1,4.5)和(1.003,1.003,4.5),结果如图4.3。

可以看到,系统可以分辨0.01m的两个点,但是当两点之间的距离达到毫米级,目标将无法分辨。
在实际应用中,目标往往是连续表面,因而各点之间会造成一定干扰。
但是有一点可以断定,目标所在位置及其周围一定范围内的置信强度,一定要大于外围没有目标的区域。
反演算法本身可行性值得认可。

  
图4.3  近距离双点反演效果图
Figure 4.3 Inversion results of two near points

4.2  预设场景的反演
如前所述,根据对隐藏空间和中介反射面的先验了解,场景反演过程可以看作一个线性反演过程。
每一个激光发射点-探测器对都会以直方图的形式输出一组时间-强度数据,只是在每个时刻的回波强度。
对于每个数据,都可以在空间中找到一系列与之对应的点,该点处存在目标的可能性与回波强度成正相关。

反演重建的关键步骤如图4.4。


 
图4.4  反演重建流程图
Figure 4.4 Flow chart of inverse and reconstruction

该算法根据各个体素权值的大小进行筛选,权值代表各个体素是目标点的可能性。
对于每组距离数据,给处于其椭球面体素赋的权值I(k)会直接影响到最终体素的选取。

4.2.1重建空间网格划分
作为反演重建的第一步,待重建空间网格的划分会对整个反演重建过程造成很大的影响。
重建结果的可视化效果将完全取决于网格划分的效果,网格划分也将决定各店之间相互干扰程度的大小。
此外,网格划分情况将决定计算量的大小和反演效率。

网格划分越细密,每组数据反演出的椭圆越薄,反演结果越精确。
但是,系统本身存在一定的延迟,数据的量化也会引入误差。
如果网格划分太过细密,部分数据的反演结果和原始目标点会存在偏差,太细的网格也会加大运算量,影响重建效率。
反之,如果网格划分太过稀疏,每个数据的反演结果都会是一个很厚的椭球面,投影到二维平面,结果会出现很强烈的圆晕效果,目标周围很大一圈的置信值也会很高。

另外,从实际操作的角度来看,网格可以均匀划分或者根据数据非均匀动态划分。

均匀划分
在计算机处理中,均匀网格划分十分容易,也便于运算。
但是,由于均匀划分会将很多不存在目标的网格引入处理,增大了运算量,重建效率受到了很大影响。

动态化网格划分
动态化网格划分,即通过数据的预处理,大体锁定目标位置,该区域的网格划分比较密集,其他区域网格划分稀疏。
该划分方法虽然加入了预处理的工作量,但可以在不影响效果的前提下提高反演效率。

4.2.2数据吻合度判定规则
反演算法的一个重要步骤是判定网格与时间-强度数据是否吻合。
由于原本的目标点与划分出的网格并不是完全对应的,在判定数据吻合程度时,需要设定一个阈值。
当网格运算距离与时间对应距离的差值小于该阈值时,认为该点处可能存在目标。
该阈值的设定会影响反演重建效果。
阈值较大时,每个真实目标点周围的多个点都会被赋予一定可能性;而阈值较小时,真实目标点可能也被滤除,无法进入最终筛选。

4.3  点云筛选方法
在上一步的反演中,我们得到了各个体素存在目标物体的可能性分布。
由于相互干扰和反演算法本身的缺陷,在目标的边缘和目标块之间的区域,可能性数值也很高,将这部分不是目标却可能性数值很大的部分称作“虚影”,需要通过一定的筛选算法将其剔除,从而得到真正的目标分布区域,提取出 轮廓信息。

4.3.1 单阈值全局筛选
对于形状单一封闭均匀的简单目标,如上文中提到的简单矩形,真实目标区域和伪目标之间的可能性数值会相差很大,可以直接设定阈值进行筛选。

而对于形状比较复杂的图形,由于强度受距离影响很大,并且在反演过程中进行了多次叠加。
因此距离和点的分布情况都会影响强度值。
单阈值的筛选可能会遗漏距离较远或者密度较小的点,影响成像结果。

4.3.2 自适应动态筛选
对于形状比较复杂的目标,单一阈值筛选不能满足成像要求,阈值太大会保留虚影,阈值太小可能会把部分目标一起筛掉。
此时需要一种动态筛选方法,根据数据特征,对空间中不同体素采取不同阈值参数。
筛选阈值选用全局最值和局部最值的叠加,对目标各区域进行自适应处理。

4.4  反演结果与影响因素分析
对3.2中设置的场景进行仿真成像,得到的结果如图4.5。
图中包括原始目标和反演结果,为了增强结果的可视化,还增加了灰度图显示并提取了成像结果的轮廓信息。

本节目标均处于xy平面上1m*1m的空间范围,距离墙面(z方向)1m,正对墙面。
扫描点为目标所在空间在墙面投影区域上的均匀采样。
得到目标在xy平面上的投影并进行筛选和轮廓提取。

 
(a) 简单方块目标成像
 
(b) 方圆双目标成像
 
(c) 人形目标成像
图4.5  场景反演结果
Figure 4.5 Results of inversion

由结果图可以看到,单一规则目标的成像效果非常明了,复杂目标则会相互干扰,造成一定的混淆。
造成这种结果的原因很多,采用人形场景,对此进行实验解释。

4.4.1扫描点数对成像的影响
当扫描点数比较少时,反演过程用到的椭球数目比较少,相邻体素被赋的权值可能会相差很大,图像圆晕效果明显,成像识别度不高。
如图4.6,当扫描点数逐渐增多,体素之间的过度越来越自然、反演图像会越来越平滑,越来越清晰。
但是,扫描点的增加,意味着实验程序会更加复杂,计算量也随之增加。
因此要找一个效果和运算量之间的折中。
其中每幅图中都包含了原始目标图,成像强度图,成像灰度图和轮廓提取图。
(a) (b) (c)(d)(e)所对应的的扫描点数分别是3*3,6*6,9*9。


 
(a) 3*3扫描成像
 
(b) 6*6扫描成像

 
(c) 9*9扫描成像
图4.6 扫描点数对成像结果影响示意图
Figure 4.6 Effects of the number of laser scannings 

在扫描点较少(如3*3)时,反演结果中的目标没有辨识度。
当扫描点数增加到一定范围(如6*6)时,成像与原始目标之间的相似性逐渐显示。
当扫描点数继续增加,成像结果几乎不会再变化。
对于该目标的成像来说,6*6是一个比较合适的折中,既保证了成像效果,又不会造成计算量的浪费。

4.4.2探测器时间分辨率对成像的影响
探测器的时间分辨率越高,得到的时间-光子数直方图越精确,反演可以叠加的椭球越多,椭球的厚度越小,反演图的圆晕效果也会相对弱化,成像越清晰。
结果如图4.7所示,(a)-(c)的分辨率依次是1dm, 1cm, 1mm。

     
  
 
图4.7 探测器时间分辨率对成像结果影响示意图
Figure 4.7 Effects of the time resolution of the detector

与第二章中的分辨率分析相呼应,即使是平面目标的非视域成像,本项研究也要求探测器要精确到厘米级甚至以下。
立体目标观察对分辨率的要求只能更高,这也是本项研究的瓶颈所在。

4.4.3体素划分对成像的影响
如4.2中分析,体素网格的划分会直接影响成像结果和系统性能,本节采用不同体素,完成仿真试验并采集结果。
如图4.8,(a)(b)(c)重建网格在xy方向的尺寸依次是1cm,5mm和1mm。


 
(a) 体素网格1cm*1cm*3mm
 
(b) 体素网格5mm*5mm*3mm
 
(c) 体素网格1mm*1mm*3mm
图4.8 体素划分对成像结果影响示意图
Figure 4.8 Effects of the voxel division

当网格偏大时,图像网格棱角和马赛克效应比较明显,但是不影响图像的整体分辨,5mm网格和1mm网格的反演结果非常接近。
因此,在可以接受的范围内,网格划分对成像结果没有明显影响。

4.4.4 扫描区域位置对成像的影响
在反演过程中,对于每组距离-时间数据,都有一个与之对应的椭球面被赋值,相邻的几组数据会相互影响。
当采样点位置正对目标时,这种影响会更多地体现在目标中心,反演结果更接近原始图像。
当采样点位置偏离目标、位于目标一侧时,靠近采样区域的目标点最终被赋的权值更大,在滤波过程中会滤掉一些原本位于目标上、但是距离采样区域很远的点,成像结果只能保留目标的一部分甚至变形无法辨认。
如图4.9所示,图中(a)(b)(c)分别是扫描位置正对目标投影、两者略有偏移和完全错开的成像结果。


 
(a) 扫描位置与目标投影位置一致
 
(b) 扫描位置与目标投影位置略有偏移
 
(c) 扫描位置与目标投影位置完全错开
图4.9 相对位置对成像结果影响示意图
Figure 4.9 Effects of the relative position

当扫描区域与目标投影完全错开时,反演算法的椭圆效应会更多地集中在目标以外区域,导致原本不存在目标的区域得到了很大的强度值。
而目标上相对独立的点,由于反演过程中叠加次数很小,得到的强度值反而很低。
反演结果很差。

4.4.5 目标与中介反射面的距离对成像的影响
在非视域成像中,时间分辨率和强度对反演过程都相当重要。
前者主要取决于探测器的时间分辨率,后者则与探测器和回波强度都有很大关系。
在第二章中,计算了回波强度的表达,可以看到,强度会随目标距离中介反射面的距离以四次方关系衰减。
在数据存储过程中,该强度被量化。

将目标与中介反射面的距离设定为1m,2m,3m,5m,,得到的结果如图4.10。


  
(a) 距离1m                         (b) 距离2m
  
(c) 距离3m                         (d) 距离5m
图4.10 距离对成像结果影响示意图
Figure 4.10 Effects of the distance

在常用的方法中,对于极弱光的探测,常用TCSPC,可以探测到单个光子。
在仿真中,强度数据以单光子能量为单位,舍尾后以整数形式存储。

当距离比较近时,回波光子数很多,舍尾造成的误差可以忽略不计。
但是当把距离拉远,回波逐渐减弱,舍尾对数据的影响将造成回波能量分布的严重失真。

当距离增加到5m时,回波不足以收集反演。
仿真结果证实了该研究在距离上的局限性。
这也是激光非视域成像技术目前依旧局限在实验室研究的原因之一。

4.5  三维目标成像结果
将目标设为三维的兔子模型,经过光线追踪后得到图3.3所示的可见部分点云,在其所在空间进行反演,得到的切片图如图4.11,切片位置在z=0, z=0.2。


 
图4.11  三维目标成像结果
Figure 4.11 Image of 3D bunny

图中可以看到,由于各点之间的相互干扰,三维反演结果几乎无法为目标的识别提供任何参考意见。
也由于这也是本项研究中需要深入思考的一个问题。

 
4.6  光锥变换反演
相比于经典反演方法,采用光锥变化方法的步骤简单,运算量小,并且在数学上是精确可解的。

反演步骤为:(1)用变换 
4.7  本章小结
本章实现了非视域空间的反演。
验证了反演算法的可行性并对3.2中设置的场景进行了反演,得到了非视域二维目标物体的大致轮廓。
分析了成像过程中的参数设置对结果的影响。
除了系统本身参数之外,观察位置和目标的相对位置会对成像结果造成很大影响。
但是在现实应用中,目标位置是未知的,因而相对位置不可预知,只能通过多个位置的观察获取更好结果。

该方法不适用于三维目标的成像,三维目标在z方向的分布会严重影响反演过程中各体素的强度取值。
这也解释了大多数文献都停留在二维目标成像的现状。

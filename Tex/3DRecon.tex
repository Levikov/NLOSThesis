\chapter{点云的三维重建}\label{chap:3DRecon}

在拍照成像的基础上,激光扫描设备更加注重物体细节,扩展了物体位置和形状的数据获取方式,使得被测物体更多细节的数据获取成为可能。
根据采样数据进行目标表面的重建,恢复表面各点之间的连接,得到尽可能平滑的表面,在许多实际应用中都具有重要价值。
在仪器制造等工业领域和日常产品的生产制造中,可以采用逆向工程提高效率和产量[47],在医学、服装等应用中都有重要价值[48]。

本章用来重建的点云数据来自斯坦福大学ply点云库,是相对均匀、比较完整、轮廓清晰的数据集。
算法采用C++编程实现。

\section{PCL点云库}
PCL(point cloud library)对于点云处理的意义,不亚于opencv在图像处理中的地位。
PCL是一个第三方C++集成库,在点云处理中的应用非常广泛。
其中集成了很多著名的点云处理算法,集成了很多高效数据结构,实现了大量点云处理相关的通用算法[49],可以解决点云获取、跟踪、配准以及曲面重建等多种问题[50]。
PCL可以在多种操作系统运行。

本文采用PCL1.6.0+vs2010的配置,完成该章节的泊松和贪婪三角形点云重建工作;由于powercrust算法基于三角网格完成,更适合于矩阵运算,该部分工作采用MATLAB R2018a进行。

\section{法向量估计方法}
由于本文所述非视域成像过程中,只能得到非视域场景中目标分布的大概位置信息,并不能得到法线信息。
因此不符合原始泊松重建方法的应用前提,需要加入对法向量的估计。

确定表面上一点的法向量方向,其实也就是一个表面相切面的估计问题,可以通过最小二乘法平面拟合的方法进行估计。
在计算法向量之前,先求该点所在平面,以附近K近邻点近似所在的局部平面表达。
若平面方程为 ,则其平面法向量为 。
其中平面拟合的最小二乘法推导如下:待拟合的点集样本为 ,偏差平方和

	  	(5.1)

当d有最小值时,平面拟合程度最高。
因此可以对d求导,进而求出极值点。
此时d为A,B,C的函数。


	  	(5.2)

求导,得

	  	(5.3)

同理,

	  	(5.4)
	  	(5.5)

由 ,得

	  	(5.6)
  
得

	  	(5.7)

其中
	  	(5.8)
	  	(5.9)

记 ,则

	  	(5.10)

\section{泊松重建}
泊松曲面重建是隐函数方法进行曲面重建的经典方法[51],利用隐性拟合的思路,通过求解泊松方程[52]并提取等值面提取,得到点云所处的表面模型[53]。
其间需要根据各点处的法向量场获得平滑指数函数。

由于本文所述非视域成像过程中,只能得到非视域场景中目标分布的大概位置信息,并不能得到法线信息。
因此不符合原始泊松重建的应用前提,需要加入对法向量的估计。

整个算法的步骤包括对具有法向量信息的输入点云信息的预处理,对全局问题离散化,对离散化后的子数据求解,求解泊松问题后的等值面提取,以及后期优化处理等。

表面重建过程:
1、定义八叉树。
使用八叉树结构存储点集,根据采样点集的位置定义八叉树,然后细分八叉树使每个采样点都落在深度为D的叶节点;
2、设置函数空间:对八叉树的每个节点设置空间函数F,所有节点函数F的线性和可以表示向量场V,基函数F采用了盒滤波的n维卷积;
3、创建向量场:均匀采样的情况下,假设划分的块是常量,通过向量场V逼近指示函数的梯度。
采用三次条样插值(三线插值);
4、求解泊松方程:方程的解采用拉普拉斯矩阵迭代求出;
5、提取等值面:为得到重构表面,需要选择阈值获得等值面;先估计采样点的位置,然后用其平均值进行等值面提取,然后用移动立方体算法得到等值面。

采用作者提供的代码和素材进行重建,结果如图5.1。
其中点云文件bunny.points.ply中,包括了362271个点的坐标和法向量信息,经过泊松重建后仅剩186924个点。

 
图5.1  泊松重建结果
Figure 5.1 Result of Poisson reconstruction 

该点云足够密集,法向量信息足够精确,因此重建出的表面比较光滑,细节保留得相当完整。
但是,对于没有法向信息、点分布也比较稀疏的模型,泊松重建的效果将大打折扣。

以Bunny模型的三个法向缺失采样点云进行重建研究,图5.2(a)到(c)是三个模型的原始形状。
模型中点的个数分别是453、8171和 35947。
经过泊松重建平滑,得到(d)(e)(f)所示结果,分别剩余155、2197和13090个顶点,构造了300、4376、26154个三角面,损失了大量信息。
只有当点云比较密集,如(c)所示时才能有细微的辨识度,但效果依旧很差。
 

 
图5.2  法向量缺失的泊松重建结果
Figure 5.2 Result of poisson reconsruction without normal vertex

泊松重建依赖于待重建表面的法向量。
在经典的泊松重建算法中,所用的ply文件中包含了各点处的法向量信息,重建效果非常好,辨识度很高。
但是,对于本文而言,各点处法向量无从得知,需要进行估算,会引入大量的误差,因此重建效果较差,不适合用于本文所述应用场合。

\section{基于贪婪投影三角形的重建}
贪婪投影三角形重建利用三维空间和二维空间之间的投影关系,将三维重建简化成了二维连接。
在重建过程中,将三维点云投影到二维平面,进行平面三角化后再连接到三维空间,有效减小了运算量[54]。

贪婪投影三角化算法的核心是从点出发,依次连接扩大成完整表面[55]。
该算法可以用来处理一个模型的多个扫描点云。
但该算法也有一定的缺陷,与泊松重建类似,它更适用于处理均匀连续的点云数据。

利用pcl进行贪婪投影三角形重建,三角化过程要局部进行,先将一个点投影到局部二维平面内,再连接其他悬空点[56],然后再进行下一点。

贪婪投影三角形重建利用三维空间和二维空间之间的投影关系,将三维重建简化成了二维连接。
在重建过程中,将三维点云投影到二维平面,进行平面三角化后再连接到三维空间,有效减小了运算量。
具体方法是先将有向点云投影到某一局部二维坐标平面内,再在坐标平面内进行平面内的三角化,最后根据平面内三位点的拓扑连接关系获得一个三角网格曲面模型。

贪婪投影三角化算法的核心是处理一系列可以使网格“生长扩大”的点(边缘点),延伸这些点直到所有符合几何正确性和拓扑正确性的点都被连上。
该算法的优点是可以处理来自一个或者多个扫描仪扫描得到并且有多个连接处的散乱点云。
但该算法也有一定的局限性,与泊松重建类似,它更适用于采样点云来自于表面连续光滑的曲面并且点云密度变化比较均匀的情况。

利用pcl进行贪婪投影三角形重建,三角化过程是局部进行的,首先沿着一点的法线将该点投影到局部二维坐标平面内并连接其他悬空点,然后再进行下一点。
需要通过以下几个函数设置参数:
1)函数SetMaximumNearestNeighbors(unsigned)和SetMu(double),控制搜索邻域大小。
前者定义了可搜索的邻域个数,后者规定了被样本点搜索其邻近点的最远距离。

2)SetSearchRadius(double),设置三角化后得到的每个三角形的最大可能边长。

3)SetMinimumAngle(double)和SetMaximumAngle(double),三角化后每个三角形的最大角和最小角。
两者至少要符合一个。

4)函数SetMaximumSurfaceAgle(double)和SetNormalConsistency(bool),这两个函数是为了把处理过程中一些不恰当的三角网格剔除,主要用于处理陡峭边缘和尖锐角以及一个表面的两边非常靠近的情况。
为了处理这些特殊情况,函数SetMaximumSurfaceAgle(double)规定如果某点法线方向的偏离超过指定角度,该点就不连接到样本点上。
该角度是通过计算法向线段之间的角度。
函数SetNormalConsistency(bool)保证法线朝向,如果法线方向一致性标识没有设定,就不能保证估计出的法线都可以始终朝向一致。
第一个函数特征值为45度(弧度)、第二个函数缺省值为false。

几个模型的重建结果如图

 
图5.3  贪婪三角形投影重建结果
Figure 5.3 Results of greedy triangle reconstruction

和泊松重建类似,点云越密集,重建效果越好。
由于脱离了对法向量的依赖,贪婪三角形法的重建效果整体要优于泊松重建。
但是在边缘和不光滑的区域,依旧存在部分三角缺失或者连线错误的情况。

\section{powercrust重建}
基于Voronoi的powercrust算法立足于三角剖分,无需了解点云的法向量,适合用于散乱点云的重建。

重建流程主要有以下几步:
计算取样点集S的Voronoi。

如果s不位于S的凸包上,令p+为Vor(S)中离s最远的点;如果s位于凸包上,取p+为凸包外“无限远”处一点,sp+的方向等于凸包上相交于s点的所有面的外法向的平均值;在Vor(S)的所有顶点中,∠p+sv的值大于π/2,选离s最远的点为p-;令P为所有极点p+和p-的集合,除了那些在无穷远处的点p+。
计算S∪V的Delaunay三角剖分。

仅留下那些三个顶点都是取样点的三角形。

除去这样的三角形T:T的法向与T的顶点到p+的向量之间的角度太大。
(与T的最大角处的顶点所成的角大于θ,与其它顶点所成的角大于3θ/2)。

连续定向三角形和极点(内部、外部),抽取出一个没有边界的分段线性流形。

powercrust重建效果展示如图5.4。


 
(a)                   (b)                   (c)
图5.4  powercrust重建结果
Figure 5.4 Results of powercrust reconstruction

\section{本章小结}
本章介绍了pcl点云库,实现了泊松和贪婪三角形投影重建,并用matlab网格化实现了powercrust重建方法。
从重建结果可以看出,针对无法给定法向量的点云数据,powercrust的重建效果最好,贪婪三角形投影次之,泊松重建效果最差。

